\documentclass{article}
\usepackage{graphicx} % Required for inserting images


\begin{document}


\begin{figure}
    \centering
    \includegraphics[width=0.6\linewidth]{../img/logo-informatica.jpeg} 
\end{figure}

\title{Programación TICS100: Ayudantía n°1}
\author{Bruno Reyes}
\date{11 de Marzo 2024}


\maketitle


\subsubsection*{1.- Diagramas de flujo}

\noindent
1.1. Crear un algoritmo que, al recibir dos valores distintos como entrada, identifique cuál de ellos es el mayor y lo muestre como resultado.\\

\noindent
1.2. Diseñe un diagrama de flujo de una calculadora simple (suma, resta, multiplicación y división) donde como input se ingresan dos números y como output el resultado de la operación escogida. \\

\subsubsection*{2.- Operadores básicos Python}

\noindent
2.1. Escriba un programa en Python que transforme la unidad de medida \textbf{Celsius} a \textbf{Fahrenheit}. Este debe tomar el dato e imprimir por pantalla el resultado.\\

\noindent
2.2. Escriba un programa en Python que transforme la unidad de medida \textbf{Kelvin} a \textbf{Fahrenheit}. Este debe tomar el dato e imprimir por pantalla el resultado.\\

\noindent
2.3. Escriba un programa que calcule la nota final \textbf{NF} del curso TICS100, si es necesario, use condicionales (if - else).  \\




\end{document}
